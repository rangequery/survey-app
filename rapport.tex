\documentclass[a4paper,12pt]{report}
\usepackage[utf8]{inputenc}
\usepackage[T1]{fontenc}
\usepackage[french]{babel}
\usepackage{graphicx}
\usepackage{hyperref}
\usepackage{listings}
\usepackage{xcolor}
\usepackage{float}
\usepackage{geometry}
\usepackage{fancyhdr}
\usepackage{titlesec}
\usepackage{enumitem}

\geometry{margin=2.5cm}

\definecolor{codegreen}{rgb}{0,0.6,0}
\definecolor{codegray}{rgb}{0.5,0.5,0.5}
\definecolor{codepurple}{rgb}{0.58,0,0.82}
\definecolor{backcolour}{rgb}{0.95,0.95,0.92}

\lstdefinestyle{mystyle}{
    backgroundcolor=\color{backcolour},   
    commentstyle=\color{codegreen},
    keywordstyle=\color{magenta},
    numberstyle=\tiny\color{codegray},
    stringstyle=\color{codepurple},
    basicstyle=\ttfamily\footnotesize,
    breakatwhitespace=false,         
    breaklines=true,                 
    captionpos=b,                    
    keepspaces=true,                 
    numbers=left,                    
    numbersep=5pt,                  
    showspaces=false,                
    showstringspaces=false,
    showtabs=false,                  
    tabsize=2
}

\lstset{style=mystyle}

\pagestyle{fancy}
\fancyhf{}
\fancyhead[L]{Application de Sondage Avancée}
\fancyhead[R]{\thepage}
\renewcommand{\headrulewidth}{0.4pt}

\begin{document}

\begin{titlepage}
    \centering
    {\huge\bfseries Application de Sondage Avancée\par}
    \vspace{2cm}
    {\Large\itshape Projet Django\par}
    \vspace{1.5cm}
    {\large Réalisé par:\par}
    {\large Benzhirou Alaeddine \& Lasri Mohamed Amine\par}
    \vspace{1cm}
    {\large Encadré par:\par}
    {\large M. Abouabid\par}
    \vfill
    {\large \today\par}
\end{titlepage}

\tableofcontents
\newpage

\chapter{Introduction}
Ce rapport présente les fonctionnalités clés de notre application de sondage en ligne, développée avec Django. Nous nous concentrerons sur deux aspects essentiels : la visualisation des données avec Chart.js et l'exportation des résultats.

\chapter{Fonctionnalités Clés}

\section{Visualisation des Données avec Chart.js}

\subsection{Intégration de Chart.js}
L'application utilise Chart.js pour créer des visualisations interactives des données de sondage. Cette intégration permet de :

\begin{itemize}
    \item Générer des graphiques dynamiques en temps réel
    \item Afficher différents types de visualisations :
    \begin{itemize}
        \item Graphiques en camembert pour les réponses catégorielles
        \item Graphiques en barres pour les comparaisons
        \item Graphiques en ligne pour les tendances temporelles
    \end{itemize}
    \item Permettre l'interaction utilisateur avec les graphiques :
    \begin{itemize}
        \item Zoom et dézoom
        \item Filtrage des données
        \item Affichage des valeurs au survol
    \end{itemize}
\end{itemize}

\subsection{Types de Visualisations}
\begin{itemize}
    \item \textbf{Réponses par Question}:
    \begin{itemize}
        \item Distribution des réponses
        \item Pourcentages et nombres absolus
        \item Comparaison avec les moyennes
    \end{itemize}
    \item \textbf{Statistiques Globales}:
    \begin{itemize}
        \item Taux de complétion
        \item Temps moyen de réponse
        \item Nombre total de participants
    \end{itemize}
\end{itemize}

\section{Système d'Exportation}

\subsection{Formats d'Export}
Le système d'exportation permet aux utilisateurs de :

\begin{itemize}
    \item \textbf{Export CSV}:
    \begin{itemize}
        \item Format universel et léger
        \item Compatible avec tous les tableurs
        \item Encodage UTF-8 pour les caractères spéciaux
    \end{itemize}
    \item \textbf{Export Excel}:
    \begin{itemize}
        \item Format .xlsx avec mise en forme
        \item Feuilles multiples pour différentes vues
        \item Formules et calculs automatiques
    \end{itemize}
\end{itemize}

\subsection{Fonctionnalités d'Export}
\begin{itemize}
    \item \textbf{Filtrage des Données}:
    \begin{itemize}
        \item Sélection des questions à exporter
        \item Filtrage par date
        \item Filtrage par type de réponse
    \end{itemize}
    \item \textbf{Personnalisation}:
    \begin{itemize}
        \item Choix des colonnes à inclure
        \item Format des dates et nombres
        \item En-têtes personnalisés
    \end{itemize}
\end{itemize}

\chapter{Conclusion}
L'application met l'accent sur l'expérience utilisateur et l'analyse des données, en offrant des outils modernes de visualisation et d'exportation. Ces fonctionnalités facilitent la prise de décision et l'exploitation des résultats de sondages.

\end{document}